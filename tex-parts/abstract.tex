\begin{abstractCZ}
    Tato bakalářská práce se primárně zaměřuje na návrh a implementaci nových komponent pro řešení stávajícího informačního systému střední průmyslové školy, s cílem eliminovat identifikované slabiny a využít potenciál pro zlepšení. Práce vychází z podrobné analýzy současného stavu systému, včetně jeho funkcionalit, toku dat, synchronizačních procesů, API rozhraní, legislativních požadavků a bezpečnostních opatření. Na základě zjištěných výsledků se práce soustředí na návrh a implementaci klíčových funkcí pro efektivnější správu školy a odbourání automatizovatelných a repetetivních úkonů. Důraz je kladen na praktickou aplikovatelnost navrhovaných řešení, jejich integraci do stávajícího systému a možnost dalšího rozvoje.
\end{abstractCZ}

\begin{keywordsCZ}
    informační systém, návrh, in-house vývoj, školství
\end{keywordsCZ}
    
    \vspace{2cm}
    
\begin{abstractEN}
    This bachelor thesis primarily focuses on the design and implementation of new components for the solution of the existing information system of a secondary industrial school, with the aim of eliminating identified weaknesses and exploiting the potential for improvement. The work is based on a detailed analysis of the current state of the system, including its functionalities, data flow, synchronization processes, API interfaces, legislative requirements and security measures. Based on the findings, the thesis focuses on the design and implementation of key features for more efficient school management and the elimination of automated and repetitive tasks. Emphasis is placed on the practical applicability of the proposed solutions, their integration into the existing system and the possibility of further development.
\end{abstractEN}
    
\begin{keywordsEN}
    information system, design, in-house development, education
\end{keywordsEN}