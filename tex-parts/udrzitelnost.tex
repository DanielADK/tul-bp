\chapter{Udržitelnost}
Udržitelnost informačního systému (IS) je klíčová pro jeho dlouhodobé úspěšné fungování a adaptaci na měnící se potřeby školy. Udržitelný IS musí být navržen tak, aby byl flexibilní, modulární a schopný integrace nových technologií a procesů. Toto zajišťuje, že systém zůstane relevantní, efektivní a schopen růstu spolu s organizací, aniž by bylo nutné jej zásadně předělávat nebo nahrazovat.

Při návrhu nových komponent IS je nezbytné přistupovat s předvídavostí a zaměřením na rozšiřitelnost. Nové komponenty by měly být navrženy tak, aby bylo možné je v budoucnu jednoduše upravovat, rozšiřovat o nové funkce nebo integrovat s dalšími systémy a technologiemi.

Základem udržitelnosti a rozšiřitelnosti IS je také kladen důraz na kvalitu kódu, dokumentaci a testování. Kvalitní, dobře dokumentovaný a otestovaný kód usnadní úpravy a rozšiřování systému. To do budoucna pomůže předcházet chybám, které by mohly vzniknout při integraci nových komponent nebo při rozšiřování funkcionalit.